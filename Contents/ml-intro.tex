% !TeX spellcheck = en_US
\chapter{Overview of Machine Learning}

A machine learning algorithm is an algorithm that has the ability to \textit{learn} from the data. A computer program is said to \textbf{learn}, if its performance at tasks in $T$, measured by $P$, improves with experience $E$ (in which the experience is equivalent to the data). \cite{goodfellow2016deep}

\section{Task $T$}
A \textit{task} is usually described by how the \ac{ML} model process a single \textit{data point}. This section presents some common \ac{ML} tasks. \cite{vu2018mlcb}

\subsection{Classification}
The task is to specify a label for the given data point. The labels are usually members of a list.

\Eg, in the problem of digit classification, the data point is images of hand-written numbers. The data set comes with their labels as well. The task is then, given a unseen image, the model would be able to tell which number is in that image. In this problem, there are 10 possible labels, \ie, $0, 1, \dots, 9$.

\subsection{Regression}

\section{Experience $E$}
\section{Performance $P$}

