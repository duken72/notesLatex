% !TeX spellcheck = en_US
\chapter{Academic Writing}
Simple and clear ideas are always better and have high number of citing. The ones that explain their approach in a simple and clear manner make it easier and more appealing for others to read their works. Thus it's easy for others to understand, apply their ideas and extend it.

\section{Examples}
Learning from examples is, without doubt, always great

\subsection{Good examples}
\begin{itemize}
	\item A well structured paper, let alone the fact that every part is also well written \cite{schaul2015prioritized}
	\begin{itemize}
		\item Introduction: explains WHAT problem they tried to address, and WHY
		\item Background
		\item Main approach
		\begin{itemize}
			\item A motivating example
			\item Explain each components and changes in their approach, also with the problems that they solve.
		\end{itemize}
	\end{itemize}
	\item \cite{andrychowicz2017hindsight}
	\item Reading a long but well-explained paper is less tiring than reading a short but complicated one \cite{silver2016mastering, silver2017mastering}
\end{itemize}

\subsection{Bad examples}
\begin{itemize}
	\item makes large claims / comments without supporting evidences
	\item examples, images are not helpful / representative / illustrative
	\item not enough examples / explanation
	\item positioning of examples / explanation
\end{itemize}
The content is great, really, I read through it, but the way it was presented is not
\begin{itemize}
	\item Learning to Sequence and Blend Robot Skills via Differentiable Optimization:
\end{itemize}
