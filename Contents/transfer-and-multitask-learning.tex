% !TeX spellcheck = en_US
\chapter{Transfer and Multi-Task Learning}
If we’ve solved prior tasks, we might acquire useful knowledge for solving a new task. In \ac{RL}, knowledge is stored in:
\begin{itemize}
	\item Q-function: tells us which actions or states are good
	\item Policy: tells us which actions are potentially useful. Some actions are never useful!
	\item Models: what are the laws of physics that govern the world?
	\item Features/hidden states: provide us with a good representation
\end{itemize}

\section{Definitions}
\begin{itemize}
	\item \textit{Transfer learning} is using experience from \textit{\textbf{one set of tasks}} for faster learning and better performance on a \textit{\textbf{new task}} in \ac{RL}.
	\item In \ac{RL}, task = \ac{MDP}
	\item \textit{Shot}: number of attempts in the target domain
	\begin{itemize}
		\item 0-shot: just run the policy trained in the source domain
		\item 1-shot: try the task once
		\item few shot: try the task a few times
	\end{itemize}
\end{itemize}

No single solution!
\begin{itemize}
	\item Forward transfer: train on one task, transfer to a new task
	\begin{itemize}
		\item Transferring visual representations \& domain adaptation
		\item Domain adaptation in reinforcement learning
		\item Randomization
	\end{itemize}
	\item Multi-task transfer: train on many tasks, transfer to a new task
	\begin{itemize}
		\item Sharing representations and layers across tasks in multi-task learning
		\item Contextual policies
		\item Optimization challenges for multi-task learning
		\item Algorithms
	\end{itemize}
	\item Transferring models and value functions
	\begin{itemize}
		\item Model-based RL as a mechanism for transfer
		\item Successor features \& representations
	\end{itemize}	
\end{itemize}

\section{Forward Transfer}
\textit{Forward transfer}: train on one task, transfer to a new task.
\begin{itemize}
	\item Transferring visual representations \& domain adaptation
	\item Domain adaptation in reinforcement learning
	\item Randomization
\end{itemize}

Common challenges:
\begin{itemize}
	\item Domain shift: representations learned in the source domain might not work well in the target domain
	\item Difference in the \ac{MDP}: some things that are possible to do in the source domain are not possible to do in the target domain
	\item Finetuning issues: if pretraining \& finetuning, the finetuning process may still need to explore, but optimal policy during finetuning may be deterministic!
\end{itemize}

\subsection{Domain Adaptation}
\hlb{Invariance assumption:} everything that is \textbf{different} between domains is \textbf{irrelevant}.

\hlb{Idea:} force a representation layer (a layer) to be domain invariant.
$p(x)$ is different, exists some $z=f(x)$ such that $p(y|z) = p(y|x)$, but $p(z)$ is the same in both source and target domains. \cite{tzeng2020adapting}

Invariance is not enough when the dynamics don't match. \cite{eysenbach2020off}
\begin{align*}
	\tilde{r}(s,a) =& r(s,a) + \Delta r(s,a)\\
	\Delta r(s_t, a_t, s_{t+1}) =& \log p_{target}(s_{t+1} | s_t, a_t) - \log p_{source}(s_{t+1} | s_t, a_t)\\
	\Delta r(s_t, a_t, s_{t+1}) =& \log p(target | s_t, a_t, s_{t+1}) - \log p(target | s_t, a_t)\\
	&- \log p(source | s_t, a_t, s_{t+1}) - \log p(source | s_t, a_t)
\end{align*}
\hlb{Problems:}
\begin{itemize}
	\item The above approach considers limits in the target domain that doesn't present in source domain.
	\item But it doesn't consider things that can happen in target domain but limited in source domain.
\end{itemize}

\subsection{Finetuning}
Challenges:
\begin{itemize}
	\item RL tasks are generally much less diverse
	\begin{itemize}
		\item Features are less general
		\item Policies \& value functions become overly specialized
	\end{itemize}	
	\item Optimal policies in fully observed \ac{MDP}s are deterministic
	\begin{itemize}
		\item Loss of exploration at convergence
		\item Low-entropy policies adapt very slowly to new settings
	\end{itemize}	
\end{itemize}

Finetuning with maximum-entropy policies: act \hlb{as randomly as possible} while collecting high rewards. \cite{haarnoja2017reinforcement}

\subsection{References}
Domain adaptation:
\begin{itemize}
	\item \citeausm{tzeng2014deep}. Deep domain confusion: Maximizing for domain invariance.
	\item \citeausm{ganin2016domain}. Domain-adversarial training of neural networks.
	\item \citeausm{tzeng2020adapting}. Adapting deep visuomotor representations with weak pairwise constraints.
	\item \citeausm{eysenbach2020off}. Off-dynamics reinforcement learning: Training for transfer with domain classifiers.
\end{itemize}

Finetuning:
\begin{itemize}
	\item \citeausm{haarnoja2017reinforcement}. Reinforcement learning with deep energy-based policies.
	\item \citeaus{andreas2017modular}. Modular multitask reinforcement learning with policy sketches.
	\item \citeaus{florensa2017stochastic}. Stochastic neural networks for hierarchical reinforcement learning.
	\item \citeausm{kumar2020one}. One solution is not all you need: Few-shot extrapolation via structured \ac{MaxEnt} \ac{RL}.
\end{itemize}

\section{Forward Transfer with Randomization}
\hlb{Idea:} If we can design the source domain, then vary it to prepare for the difficult in target domain.

\begin{itemize}
	\item Randomizing physical \ac{param} \cite{rajeswaran2016epopt}
	\item Explicit system identification \cite{yu2017preparing}
	\item Randomization for real-world control \cite{sadeghi2016cad2rl}
\end{itemize}

\subsection{References}
\begin{itemize}
	\item \citeausm{rajeswaran2016epopt}. EPOpt: Learning robust neural network policies using model ensembles.
	\item \citeausm{yu2017preparing}. Preparing for the unknown: Learning a universal policy with online system identification.
	\item \citeaus{sadeghi2016cad2rl}. CAD2RL: Real single-image flight without a single real image.
	\item \citeausm{tobin2017domain}. Domain randomization for transferring deep neural networks from simulation to the real world.
	\item \citeaus{james2017transferring}. Transferring end-to-end visuomotor control from simulation to real world for a multi-stage task.
	\item \citeausm{bousmalis2018using}. Using simulation and domain adaptation to improve efficiency of deep robotic grasping.
	\item \citeausm{rao2020rl}. RL-CycleGAN: Reinforcement learning aware simulation-to-real.
\end{itemize}

\section{Multi-Task Transfer}
\hlb{Definition:} train on many tasks, transfer to a new task.
\begin{itemize}
	\item Sharing representations and layers across tasks in multi-task learning
	\item Contextual policies
	\item Optimization challenges for multi-task learning
	\item Algorithms
\end{itemize}

Multi-task \ac{RL} corresponds to a single-task \ac{RL} in a \hlb{joint \ac{MDP}}.

\hlb{Challenges:}
\begin{itemize}
	\item \hlb{Gradient interference:} becoming better on one task can make you worse on another\\
	\textit{Negative transfer} implies the transfer actually hurting the task's performance.
	\item \hlb{Winner-take-all problem:} imagine one task starts getting good algorithm is likely to prioritize that task (to increase average expected reward) at the expensive of others.
\end{itemize}

The two approaches is not really multi-task transfer. Not really fasten the learning speed of other tasks
\subsection{Actor-Mimic}
\begin{itemize}
	\item Train multiple single-task policy, then use supervised learning to combine them into one single policy. \cite{rusu2015policy}
	\item Analogous to guided policy search (\secref{sec:guided-policy-search}), but for transfer learning. \cite{levine2013icml}
\end{itemize}

\subsection{Policy Distillation}
Divide and Conquer does has some transfer between central policy and local policies, but not \hlb{much}. \cite{parisotto2015actor}

\subsection{Contextual Policies}
Can do (discern) multiple things in the same environment.
\begin{itemize}
	\item Standard policy: $\pi_\theta(\textbf{a|s})$
	\item Contextual policy: $\pi_\theta(\textbf{a|s}, \omega)$
	\item Augmented state space: $\tilde{\textbf{s}} = \begin{bmatrix}
		\textbf{s}\\
		\omega
	\end{bmatrix} \qquad \tilde{\mathcal{S}} = \mathcal{S} \times \Omega$
\end{itemize}

\section{Transferring Models and Value Functions}
\hlb{Assumption:}
\begin{itemize}
	\item the \textbf{dynamics} is the same in both domains
	\item but the \textbf{reward function} is different
\end{itemize}

\Eg:
\begin{itemize}
	\item Autonomous car learns how to drive to a few destinations, and then has to navigate to a new one.
	\item A kitchen robot learns to cook many different recipes, and
\end{itemize}

\subsection{Transferring Models}
Due to distribution shift, zero-shot might not always work. Although the dynamics is the same, they could be in different contexts: things present in source domain might not be present in target domain, and vice versa.

Unless the samples in the source domain is broad enough that they covers target domain's. 

\subsection{Transferring Value Functions}
Value functions couple dynamics, rewards, and policies, but in linearity.
\begin{equation}
	Q^\pi(\textbf{s,a}) = \underbrace{r(\textbf{s,a})}_{\textstyle \text{rewards}} + \gamma \mathbb{E}_{\textbf{s}' \sim \underbrace{p(\textbf{s}'|\textbf{s,a})}_{\textstyle \text{dynamics}}, \textbf{a}'\sim \underbrace{\pi(\textbf{a}'|\textbf{s}')}_{\textstyle \text{policies}}} [Q^\pi (\textbf{s}', \textbf{a}')]
\end{equation}

Let $\textbf{P}^\pi \textbf{v}$ denote a vector \textbf{w} of length $|S||A|$:
\begin{align}
	\textbf{w}(\textbf{s,a}) &= \mathbb{E}_{\textbf{s}' \sim p(\textbf{s}'|\textbf{s,a}), \textbf{a}'\sim \pi(\textbf{a}'|\textbf{s}')} [\textbf{v}(\textbf{s}', \textbf{a}')]\\
	\textbf{w}_{|S||A|\times 1} &= \textbf{P}^\pi_{|S||A|\times|S||A|} \textbf{v}_{|S||A|\times 1}\\
	Q^\pi &= r + \gamma \textbf{P}^\pi Q^\pi\\
	Q^\pi &= (\textbf{I} - \gamma \textbf{P}^\pi)^{-1} r
\end{align}

Let:
\begin{itemize}
	\item $\phi$ be an $|S||A| \times N$ \textit{feature} matrix
	\item $\psi$ be an $|S||A| \times N$ matrix such that $\psi = (\textbf{I} - \textbf{P}^\pi)^{-1}\phi$
	\item if $r=\phi w$, then $Q^\pi = \psi w$, with $w$ is a $1 \times N$ row vector\\
	In other words, if $r$ is a linear combination of $\phi$, then $Q^\pi$ is a linear combination of $\psi$
	\item $\psi_i$ is a "successor feature" for $\phi_i$
	\item for any \textit{new} reward function, if we can fit $r \approx \phi w$, we get $Q^\pi \approx \psi w$\\
	\note this holds for $Q^\pi$, not $Q^*$
	\[ Q^*(\textbf{s,a}) = r(\textbf{s,a}) + \gamma \mathbb{E}_{\textbf{s}' \sim p(\textbf{s}'|\textbf{s,a})}\left[ \underset{\textbf{a}'}{\max} Q^\pi(\textbf{s}', \textbf{a}') \right] \qquad \text{(no longer linear)}\]
\end{itemize}

\todo{} how to find $\psi, \phi$ given $\textbf{P}^\pi$

\hlb{Simplest use:} evaluation
\begin{enumerate}
	\item get small amount of data $(\textbf{s}_i, \textbf{a}_i, r_i, \textbf{s}_i')$ in new \ac{MDP}
	\item fit $w$ such that $\phi(\textbf{s}_i, \textbf{a}_i)w \approx r_i$ (linear regression)
	\item initialize $Q^\pi(\textbf{s,a}) = \psi(\textbf{s,a}) w$
	\item finetune $\pi$ and $Q^\pi$ with any \ac{RL} method
\end{enumerate}

\hlb{More sophisticated use:} train multiple $\psi^{\pi_i}$ functions for different $\pi_i$
\begin{itemize}
	\item choose initial policy $\pi(\textbf{s,a}) = \underset{\textbf{a}}{\arg\max} \psi^{\pi_i}(\textbf{s,a}) w$
	\item this provides a better initial policy in general
\end{itemize}

\hlb{Additional notes:} successor representations with $\phi = \textbf{I}$ \dots \cite{dayan1993improving}