% !TeX spellcheck = en_US
\chapter{CS Algorithms}

\section{Sorting}

References:
\begin{itemize}
	\item \href{https://youtu.be/OOBBI-kSChM}{BATTLE OF THE SORTS: which sorting algorithm is the fastest? (visualization)}
	\item \href{https://stackoverflow.com/q/5038895/11397588}{stackoverflow}
	\item \href{https://www.cprogramming.com/tutorial/computersciencetheory/sortcomp.html}{cprogramming.com}
\end{itemize}

Sorting algorithms:
\begin{itemize}
	\item Selection sort: finds the next smallest element
	\item Bubble sort: swaps adjacent out-of-order elements
	\item Insertion Sort: inserts next element into its place
	\item Heap Sort: Selection sort with a heap
	\item Merge Sort: Divide and conquer method
\begin{lstlisting}[language=C++]
void merge(
int a[], const int left, const int mid, const int right);	

void mergeSort(int a[], const int begin, const int end)
{
	if (begin >= end)
	return;
	
	int mid = begin + (end - begin) / 2;
	mergeSort(a, begin, mid);
	mergeSort(a, mid + 1, end);
	merge(array, begin, mid, end);
}
\end{lstlisting}
	\item Timsort: A hybrid version of Insertion and Merge sort.
	\item Quick Sort: swaps elements around a pivot
	\item IntroSort: A hybrid version of Insertion, Quick and heap sort
\end{itemize}

\section{Backtracking}
\begin{itemize}
	\item Backtracking can be defined as a general algorithmic technique that considers searching every possible combination in order to solve a computational problem.
	\item Backtracking uses recursive calling to find the solution by building a solution step by step increasing values with time.
	\item It removes the solutions that doesn't give rise to the final solution of the problem based on the given constraints.
\end{itemize}

\section{Dynamic Programming}
\begin{itemize}
	\item Dynamic Programming is an optimization over plain recursion.
	\item It is the approach of breaking down a problem into simpler and repeated sub-problems.
	\item The solutions to each sub-problem are stored so that each of them is only solved once.
\end{itemize}

\href{https://youtu.be/aPQY__2H3tE}{Steps:}
\begin{itemize}
	\item Visualize examples
	\item Find suitable sub-problem
	\item Find relationships among sub-problems
	\item Generalize the relationship
\end{itemize}